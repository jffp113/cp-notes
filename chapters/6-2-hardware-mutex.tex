\section{Solving Mutual Exclusion - Hardware Level}

\subsection{The \texttt{test\&set} \& \texttt{reset} primitives}

Consider that both primitives are atomic and $X$ is a shared register initialized to $1$.
\begin{itemize}
    \item \texttt{test\&set} will set $X$ to $0$ and return the previous value.
    \item \texttt{reset} will write $1$ to $X$
\end{itemize}

\begin{lstlisting}[
    caption=Acquire Mutex Operation.
]
fn acquire_mutex(i) {
    while (r != 1) {
        r = X.test&set();
    }
}
\end{lstlisting}

\begin{lstlisting}[
    caption=Release Mutex Operation.
]
fn release_mutex(i) {
    X.reset();
}
\end{lstlisting}

The implementation is based on the following invariant:

\begin{equation*}
     X + \sum_{i=1}^{n} r_i = 1
\end{equation*}

\subsection{The \texttt{swap} primitive}

Consider a shared register $X$,
the \texttt{swap(v)} primitive will assign $v$ to $X$
and return the previous value of X.

We can think of the previous instructions,
\texttt{test\&set} and \texttt{reset} as \texttt{swap} operations with fixed values,
respectively, \texttt{swap(0)} and \texttt{swap(1)}.

\begin{lstlisting}[
    caption=Acquire Mutex Operation.
]
fn acquire_mutex(i) {
    r = 0;
    while (r != 1) {
        r = X.swap(r);
    }
}
\end{lstlisting}

\begin{lstlisting}[
    caption=Release Mutex Operation.
]
fn release_mutex(i) {
    X.swap(r);
}
\end{lstlisting}

The invariant from \texttt{test\&set} applies to \texttt{swap} as well.
Since this approach is just a re-implementation of the previous ones with a different instruction,
it offers the same guarantees.

\subsection{The \texttt{compare\&swap} primitive}

Let $X$ be a shared register and \texttt{old} and \texttt{new} be two values.
The \texttt{compare\&swap} primitive returns a boolean value and may be defined as follows:

\begin{lstlisting}[caption=\texttt{compare\&swap} implementation.]
fn compare&swap(old, new) {
    if (X == old) {
        X = new;
        return true;
    } else {
        return false;
    }
}
\end{lstlisting}

The code above is considered to be executed atomically.

Consider $X$ to be an atomic register initialized to $1$,
implementing mutual exclusion with \texttt{compare\&swap} we have.

\begin{lstlisting}[
    caption=Acquire Mutex Operation.
]
fn acquire_mutex(i) {
    while (r != 1) {
        r = X.compare&swap(1, 0);
    }
}
\end{lstlisting}

\begin{lstlisting}[
    caption=Release Mutex Operation.
]
fn release_mutex(i) {
    X = 1;
}
\end{lstlisting}

This and the previous algorithms are not starvation free.

\subsection{Deadlock \& Starvation Free Algorithm}

\begin{lstlisting}[
    caption=Acquire Mutex Operation.
]
fn acquire_mutex(i) {
    FLAG[i] = up;
    wait(TURN == i || FLAG[TURN] == down);
    LOCK.acquire_lock(i);
}
\end{lstlisting}

\begin{lstlisting}[
    caption=Release Mutex Operation.
]
fn release_mutex(i) {
    FLAG[i] = down;
    if (FLAG[TURN] == down) {
        TURN = (TURN % n) + 1;
    }
    LOCK.release_lock();
}
\end{lstlisting}

While this approach provides mutual exclusion, progress and no starvation,
it does not provide fairness, meaning that we may wait for a long time until we acquire the lock.

\subsection{The \texttt{fetch\&add} primitive}

Let $X$ be a shared register, the primitive \texttt{fecth\&add} will atomically increment the register $X$ and return the new value.
There are a couple of variants of this primitive in which the value return is the previous or the incremented value is not $1$ but rather passed as an argument.