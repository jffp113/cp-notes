\section{Parallel Programming}

Parallel programs often start as sequential programs.
This approach has the advantage of having a previous codebase to validate the new approach,
furthermore it is easier to start with a sequential approach given the ease of programming and debugging.

To parallelize a program we first have to evaluate if it is worth the effort,
to do so we first identify the hotspots with a profiler.

The parallelization process starts with the hotspots,
applying small changes with testing in-between.

\subsection{Finding Concurrency}

The first step in parallelizing a candidate program is to find concurrency,
in other words, we have to find tasks that can be executed at the same time.

To do so, we need to consider flexibility, efficiency and simplicity.

\subsubsection{Task Decomposition Guidelines}

\paragraph{Flexibility}
The program design should afford flexibility in the number and size of the tasks generated.

\paragraph{Efficiency}
Tasks should have enough work as to amortize the cost of creating and managing them.
furthermore, they should be sufficiently independent so that managing dependencies does not become the bottleneck.

\paragraph{Simplicity}
The code must remain as simple and readable as possible so as to be debuggable and quick to understand and modify.

\subsubsection{Data Decomposition Guidelines}

Often implied by task decomposition,
data decomposition is a good starting point when the main computation is organized around manipulation of large data structures
or similar operations are applied to different parts of the data structure.

\paragraph{Flexibility}
The size and number of data chunks should support a wide range of executions.

\paragraph{Efficiency}
Data chunks should generate considerable amounts of work to minimize the communication and management impact.

\paragraph{Simplicity}
Complex data compositions can get difficult to manage and debug.

\subsection{Algorithmic Structure Design Space}

The high level approach of an algorithm can be split in to three possibilities, all approaches can then be split into two sub approaches,
the linear or recursive approach.

\begin{itemize}
    \item Organize by Tasks - the linear approach to this problem is to make use of task parallelism, the recursive approach is through divide and conquer.
    \item Organize by Data Decomposition - the linear approach of data decomposition is by making use of geometric decomposition (arrays), the recursive approach is through recursive data structures.
    \item Organize by Data Flow - the linear solution is to make use of the pipeline pattern, the recursive approach is based on event coordination.
\end{itemize}

\subsection{Implementation Environment}

\subsubsection{Program Structures}

\paragraph{Single Program Multiple Data}
In the SMPD approach all tasks execute the same program in parallel but each has its own set of data.

\paragraph{Master/Worker Pattern}
The master/worker is works by having a thread distribute work amongst the others,
the workers execute concurrently and each worker takes tasks from the bag of tasks.
It is suited for embarrassingly parallel problems.

\paragraph{Loop Parallelism Pattern}
This pattern exploits the iterative nature of some programs by parallelizing the execution of loop iterations.
This requires the loop not to have dependencies among iterations.

\paragraph{Fork/Join Pattern}
A main task forks off some number of other tasks that then continue in parallel, the main task then waits for completion before moving on.

\paragraph{Pipeline Pattern}
Tasks are applied in sequence to data.
