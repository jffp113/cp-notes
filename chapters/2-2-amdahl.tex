\section{Amdahl's Law}

Amdahl's law fixes the problem size and varies the number of processors,
hence the denomination of Fixed Size Speedup.
The law relates at the reduction of the execution time.
Let $f$ be the sequential fraction of a program, $1-f$ is the part that can be parallelized.

\begin{equation}
    \begin{split}
        S_p & \le \frac{T_1}{T_p} = \frac{T_1}{(f \cdot T_1) + \frac{(1_f)T_1}{p}}\\
        S_p & \le \frac{1}{f + \frac{1-f}{p}}\\
        S_{p \leadsto \infty} & \le \frac{1}{f}
    \end{split}
\end{equation}

\paragraph{Scalability}
When considering scalability, Amdahl's law refers to the abillity of a parallel algorithm to achieve performance gains proportional to the number of processors and the size of the problem.

\paragraph{Application}

Amdahl's law is applicable under the following circumstances:
\begin{itemize}
    \item When the problem size is fixed.
    \item Strong scaling ($p\rightarrow\infty$, $S_p = S_{\infty} \rightarrow \frac{1}{f}$).
    \item Speedup bound is determined by the degree of sequential execution time in the computation.
\end{itemize}

\paragraph{Example}

If $90\%$ of the computation can be parallelized, what is the maximum speedup achievable using 8 processors?

We first calculate the sequential part of the program:

\begin{equation*}
    \begin{split}
        0.9 & = 1 - f\\
        f & = 0.1
    \end{split}
\end{equation*}

And then apply Amdahl's law:

\begin{equation*}
    \begin{split}
        S_8 & \le \frac{1}{0.1 + \frac{1-0.1}{8}}\\
        S_8 & \le 4.7
    \end{split}
\end{equation*}
