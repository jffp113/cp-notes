\section{Parallel Models \& Dependencies}

\subsection{Parallelism, Correctness \& Dependencies}

Parallel execution shall always be constrained by the sequence of operations needed to be performed for a correct result.
Such execution must address control, data and system dependencies.

A \textit{dependency} arises when one operation depends on an earlier operation to complete and produce a result before this later operation can be performed.

\subsubsection{Sequential Consistency}

Sequential consistency implies that the execution of two statements do not interfere with each other, meaning their running order is irrelevant.
This means that the statements are \textit{independent},
if the order of execution affects the computation outcome they are deemed \textit{dependent}.

\subsubsection{Dependencies}

\paragraph{True Dependency}
A statement $S2$ has a true dependency on statement $S1$ if and only if $S2$ reads a value written by $S1$.
This dependency is also known as Read After Write (RAW).

Formally we can describe a true dependency as follows:
\begin{equation*}
    out(S_1) \cap in(S_2) \neq \emptyset
\end{equation*}

\paragraph{Anti-Dependency}
A statement $S2$ has an anti-dependency in statement $S1$ if and only if $S2$ writes a value read by $S1$.
This dependency is also known as Write After Read (WAR).

Formally we can describe an anti-dependency as follows:
\begin{equation*}
    in(S_1) \cap out(S_2) \neq \emptyset
\end{equation*}

\paragraph{Output Dependency}
A statement $S2$ has an output dependency on $S1$ if and only if $S2$ writes a variable written by $S1$.
This dependency is also known as Write After Write (WAW).

Formally we can describe an output dependency as follows:
\begin{equation*}
    out(S_1) \cap out(S_2) \neq \emptyset
\end{equation*}

\paragraph{Loop-Carried Dependencies}
A loop-carried dependency is a dependency between two statement instances in two different loop iterations.