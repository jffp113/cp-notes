\section{Gustafson-Barsis’ Law}

\newthought{...speedup should be measured by scaling the
    problem to the number of processors, not by fixing
    the problem size. - John Gustafson}

Also denominated as Scaled Speedup, it is interested in larger problems when scaling.
In other words, for the same time, how much more work can we do?

Considering $a$ the non-parallelizable part and $b$ as the parallelizable part,
we calculate $T_P$, for $P$ processors, as follows:
\begin{equation}
    \begin{split}
        T_P & = a + P \cdot b\\
        T_1 & = a + 1 \cdot b\\
        & = a + b
    \end{split}
\end{equation}

Given that the wall-clock execution time is always the same,
the scaled speedup is calculated on the volume of data processed.

\begin{equation}
    \begin{split}
        S_P & \le \frac{T_P}{T_1}\\
        & \le \frac{a + P \cdot b}{a + b}
    \end{split}
\end{equation}

Consider $\alpha = \frac{a}{a+b}$ as the sequential fraction of the parallel execution time.
We can then define the scaled speedup as follows:

\begin{equation}
    \begin{split}
        S_P & \le \alpha + P \cdot (1-\alpha)\\
        & \le P - \alpha \cdot (P-1)
    \end{split}
\end{equation}

\paragraph{Scalability}
The ability of a parallel algorithm to achieve performance gains proportional to the number of processors and the size of the problem.

\paragraph{Application}

The Gustafson’s law applies under the following scenarios:
\begin{itemize}
    \item When the problem size can increase.
    \item When the number of processors increases.
    \item Speedup function include the number of processors.
    \item Can maintain or increase parallel efficiency as the problem scales.
\end{itemize}

\paragraph{Example} An application executing on $64$ processors spends $5\%$ of the total time on non-parallelizable computations.
What is the scaled speedup?

\begin{equation}
    \begin{split}
        S_64 & \le P - \alpha \cdot (P-1)\\
        & \le 64 - 0.05 \cdot (64 - 1)\\
        & \le 60.85
    \end{split}
\end{equation}