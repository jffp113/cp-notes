\subsection{Lock-Free Synchronization}

\subsubsection{Progress Conditions}
\paragraph{Obstruction Freedom}
At any point a single thread executed in isolation will complete its operation in a bounded number of steps.
All lock-free algorithms are obstruction-free.

This approach is the simplest one however it is too weak to guarantee progress.

\paragraph{Lock Freedom}
When the program threads are run sufficiently long, at least one of the threads makes progress
(for a sensible definition of progress).

In most cases it is strong enough,
while more complex than an obstruction free approach it is simpler than wait free approaches.
With limited contention it will behave as wait free.

\paragraph{Wait Freedom}
Every operation has a bounded number of steps the algorithm will take before the operation completes.
Being the stronger approach it is also the most complex.