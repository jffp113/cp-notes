\subsection{Data Races}
Code is supposed to be executed atomically since there are multiple dependent instructions manipulating the same data.
This creates the possibility to several possible interleavings.

\begin{lstlisting}[caption=Deposit Operation.]{}
void Bank::Deposit(int amount)
{
  int balance = getBalance();
  setBalance(balance + amount);
}
\end{lstlisting}

\begin{lstlisting}[caption=Withdraw Operation.]{}
void Bank::Withdraw()
{
  int balance = getBalance();
  setBalance(balance - amount);
}
\end{lstlisting}

As we can see in the above example,
if the code is run in different threads the withdraw may not be reflected on the final balance,
this happens because a thread (Thread $1$) may read the balance,
then the other thread (Thread $2$) will update it,
when Thread $1$ continues the balance value it has is now outdated.

\subsubsection{Data Race Detection}

Data race detection may be executed by static or dynamic analysis.
Static analysis depends on compiler features and may report false positives.
Dynamic analysis happens at runtime, all reported data races, are data races.

\paragraph{Happens-Before Algorithm}
The Happens-Before algorithm defines a partial order for events in a set of concurrent threads.
In a single thread, happens-before reflects the temporary order of event occurrence.
Between threads, $A$ happens before $B$ if $A$ is an unlock access in one threads
and $B$ is a lock access in a different thread\footnote{Data races between threads are possible if accesses to shared variables are not ordered.}.

\begin{equation}
  \mathnormal{if}\left(a = \mathtt{unlock(mu)} \wedge b = \mathtt{lock(mu)}\right) \Rightarrow a \rightarrow b
\end{equation}

\paragraph{Lock-Set Algorithm}
The algorithm may fail and the program still be data race free (false positives),
this happens since the algorithm check a sufficient condition for data-race freedom.
This algorithm does not check for data races but rather for a consistent lock discipline.
The algorithm defines two data structures, \texttt{LocksHeld(t)}, the set of locks currently held by the thread $t$ and
\texttt{LockSet(x)}, the set of locks that could potentially be protecting variable $x$.
\begin{itemize}
  \item When thread $t$ acquires lock $l$ we have:$$\mathtt{LocksHeld(t)} = \mathtt{LocksHeld(t)} \cup \{l\}$$
  \item When thread $t$ releases lock $l$ we have:$$\mathtt{LocksHeld(t)} = \mathtt{LocksHeld(t)} \setminus \{l\}$$
  \item When thread $t$ accesses location $x$ we have:$$\mathtt{LockSet(x)} = \mathtt{LockSet(x)} \cap \mathtt{LocksHeld(t)}$$
\end{itemize}
A data race is detected if \texttt{LockSet(x)} becomes empty.