\subsection{Algorithms, Programs \& Processes}

\paragraph{Sequential Algorithm}
A sequential algorithm is a forma description of the behavior of a sequential state machine.

\paragraph{Concurrent Algorithm}
The description of a set of sequential states machines that cooperate through a communication medium is called a concurrent algorithm.

\paragraph{Program}
When the algorithm is written in a specific programming language.

\paragraph{Process}
The running instance of an algorithm and thus of a program as well.

\subsection{Process Synchronization}
Process synchronization occurs when the progress of one or several processes depends on the behavior of other processes.
Two types of process interaction require synchronization, competition and cooperation.

\subsubsection{Competition}
Competition occurs when processes compete to execute some statements and only one process at a time is allowed to execute them.

\subsubsection{Cooperation}
Cooperation occurs when one process can only progress after some event on another process.

\paragraph{Barrier}
A barrier is a set of control points, one per process involved in the barrier,
such that each process is allowed to pass its control point only when all other processes have attained their own control points.
From an operation point of view, each process has to stop until all other processes have arrived at their control point.

\subsubsection{The Producer/Consumer Problem}

In the producer/consumer problem we have a producer that loops forever, producing data items,
and a consumer that loops forever, consuming said data items.

It is required to ensure that,
only produced data items are consumed and each produced item is consumed exactly once.

\paragraph{Synchronization Barrier}
One approach to the problem consists in using a synchronization barrier to ensure that only when a data item is produced the consumer is able to consume it.
While this approach works it is far from efficient.

\paragraph{Shared Buffer}
The buffer has size greater than 1 and the underlying structure can be a queue or a circular array.
Whenever the producer has a new item, it adds the item at the end of the structure,
the consumer then withdraws items from the head of the structure.

This way the producer only waits whenever the structure is full,
the consumer will only wait whenever the structure is empty.

\subsubsection{The Mutual Exclusion Problem}

\paragraph{Critical Section}
A part (or several parts) that are required to be executed by a single process at a time.

\paragraph{Solution}
We are required to provide an entry algorithm as well as an exit algorithm,
these are used to respectively enter and exit the critical section,
ensuring that the critical section is only executed once.

\paragraph{Definition}
The mutual exclusion problem consists in implementing the operations to acquire and release a mutex
in such a way that the following properties are always satisfied:

\begin{itemize}
    \item \textbf{Mutual Exclusion} - at most one process at a time executes the critical section code.
    \item \textbf{Starvation-Freedom} - each invocation of \texttt{acquire\_mutex} for any process $p$, made by $p$ eventually terminates.
\end{itemize}